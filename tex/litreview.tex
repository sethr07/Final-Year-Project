%%litreview
\label{relevantlit}
Henniger et al.~\cite{180213} perform Internet-wide scans of the entire IPv4 address space for hosts that listen on port 22 (SSH) or 
port 443 (HTTPS) using the NMap tool. NMap is an open-source network scanner that is used for network audits. They perform TLS or SSH handshakes
on hosts that listen on port 443 or port 22, respectively, and gather most of the data that might help discover weak keys. 
For TLS, they capture most of the metadata like certificate information and other X.509 certificate fields. For SSH, they collect the host keys 
using a simple client developed in C. All of the scans and data processing were carried out using Amazon's EC2 service, which provided them with 
an infrastructure to collect and analyse data efficiently. After processing the information, they identify some patterns between 
a host that shares keys and try to pinpoint some of the causes behind these. 
In the TLS protocol, the server sends its public key in a TLS certificate during the handshake stage. This key is 
used to provide a signature during the handshake stage or can be used to encrypt session details depending on whether RSA or DSA encryption algorithm is 
negotiated between the client and the server. Similarly, in the SSH protocol, the host key allows the server to authenticate itself to the client by providing a signature during 
the handshake stage. In both these protocols, if the private keys are known by an attacker and depending on the type of the 
encryption scheme (RSA or DSA), an attacker can perform different attacks like man-in-the-middle or decrypt the message containing
the session key and, in turn, use that to decrypt the entire session.\\
\cite{180213} further computes the private keys using the weak public keys by exploiting the weakness of RSA and DSA algorithms when used 
with insufficient randomness. As a result, they can compute private keys for 0.50\% of the total TLS hosts and 1.06\% of the SSH hosts found from their 
respective public keys since the key pair is related mathematically.  
They scanned 12.8 million TLS hosts and only obtained 5.8 million unique certificates for them. For SSH, they scanned 10.2 million hosts 
and obtained 6.2 million unique keys. Indicating that about 61\%  and 65\% of the TLS and SSH hosts shared the same key as another host 
in the scans. Investigating the causes of this key sharing between hosts, they found that some of the key sharing was not 
due to vulnerabilities in the infrastructure but large hosting providers that used a single key for multiple IP addresses. Another significant 
reason was TLS certificates that belonged to the same organisation. Therefore, they excluded exceptions from their data analysis process and clustered the 
remaining host that shared keys.\\\\
Dr.~Farrell's work on surveying long-term cryptographic keys for SSH, mail, and web protocols is relevant literature as this project builds 
directly from~\cite{cryptoeprint:2018:299}. The research surveys keys for ten countries and describes scans that involved hosts 
in Ireland and Estonia. The program~\cite{sftcdsur24:online} used to carry out these scans made use of Censys databases in 2017 and Maxmind databases in 2018 
to perform these scans. Then ZMap and ZGrab are used to collect data on hosts that listen on port 25, and further, the data is processed and analysed.
From one of the scans in Ireland carried out in 2018 that involved 18,268 hosts that offer at least one SSH or TLS service, approximately 
53\% hosts shared keys. Out of the 54,447 host-port combinations, only 36\% unique keys were seen through that scan. The research also reports 
key sharing among countries and shows how widespread this resue is. The clustering of hosts is based on the fingerprint information collected from 
SSH hosts and hosts that use TLS on top of the standard protocols. ``If any two IPs share a fingerprint irrespective of the port, they are clustered 
together'' to check for key reuse among hosts. Other metadata during TLS and SSH handshakes are collected for extensive data analysis. 
After the data analysis, some scripts are used to visualise the cluster information to communicate with local asset holders to understand better the causes 
of this key reuse from an asset holder's point of view and how the network security infrastructure can be improved. 
\cite{cryptoeprint:2018:299} introduces a metric called HARK \%: ``Host that is resuing keys'' for quantifying this key reuse. One of the main 
hypotheses of the work was to see if there is a correlation between the HARK\% and improvement in the security posture.

\noindent As outlined in section~\ref*{resobjectives}, one of the goals of this project is to migrate the code to Python3, refactor it to increase readability, and 
carry out some optimisation to reduce the memory overhead. This is critical since most of the work carried out in~\cite{cryptoeprint:2018:299} was done using a combination of Python2 and Bash Scripts, is quite memory intensive, 
and Python2 has deprecated since. The project also checks how key reuse has evolved over the years since 2018. 

\section{Summary}
Summarising the section above, there is widespread key reuse in a vulnerable due to a variety of reasons. 
Although a single entity cannot be blamed, regular scanning might help in better understanding some of the causes and dangers behind the 
key reuse from both an asset holder and a user point of view. 
\noindent This section summaries some of the causes and dangers of this key reuse as described in~\cite{180213,cryptoeprint:2018:299}.   
\subsection{Causes of Key Reuse}
\begin{itemize}
    \item \textbf{Hardcoded Keys:} Many devices are shipped with default keys by manufacturers in the firmware of the device. This contributes
    to key reuse as some users might not have the knowledge on how to change these keys.
    \item \textbf{Lack of Randomness:} A significant reason for key reuse was the lack of entropy during key generation. 
    This might be due to faulty random number generators or due to high costs of computations. 
    \item \textbf{Multihomed Hosts:} If a host has more than one network interface with different IPv4 addresses, it will show as individual 
    hosts in the scans. 
    \item \textbf{Virtual Machines:} If a few hosts share a key and all of those hosts are VMs, that will contribute to this. However, 
    in this case, it can be passed as acceptable as all hosts are under one physical entity.
\end{itemize}
\subsection{Dangers of Key Reuse}
\begin{itemize}
    \item \textbf{Cross Protocol Attacks:} Even though new versions of TLS are available, the migration from the older version is still slow. 
    Key reuse can increase the probability of cross-protocol attacks, and old versions of TLS/SSL may be vulnerable to these. 
    \item \textbf{Masquerading:} If asymmetric encryption is used for authentication, then key reuse may enable an attacker to pose as a host if 
    a breach of a host occurs in a cluster. 
    \item \textbf{Credential Exposure:} If sensitive information like passwords are sent using plain text protocols like SMTP, and if an 
    active attacker can masquerade as a host in a cluster, they can capture those credentials. 
    \item \textbf{Cost of Leaks:} Since private keys are usually stored locally, there is always some probability of leaks due to hardware failures. 
    If there is a leak from a host that shares keys with other hosts, this will affect all of them. Moreover, as the cluster size increases, the impact 
    of these leaks would also increase. 
\end{itemize}